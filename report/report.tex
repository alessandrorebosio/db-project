\documentclass[a4paper,11pt]{report}

\usepackage{alltt, fancyvrb, url}
\usepackage{graphicx}
\usepackage[utf8]{inputenc}
\usepackage{float}
\usepackage[table]{xcolor}
\usepackage{hyperref}
\usepackage{tabularx}
\usepackage[top=2cm, bottom=2cm, left=2cm, right=2cm]{geometry}
% \usepackage{showframe}

% Questo commentalo se vuoi scrivere in inglese.
\usepackage[italian]{babel}

\usepackage[italian]{cleveref}

\title{Elaborato per il corso di\\''Basi di Dati''}
\author{
    Maisam Noumi\\
    Alessandro Rebosio (alessandro.rebosio@studio.unibo.it)\\
    Filippo Ricciotti
}
\date{\today}


\begin{document}


\maketitle

\tableofcontents

\chapter{Analisi dei requisiti}
Si ha come obiettivo la realizzazione di un database per la gestione di un agriturismo che offre un'ampia gamma di servizi: ristorazione,
ospitalità, piscina, attività con animali, sport e molto altro.

\section{Intervista}
Il sistema che si vuole realizzare è finalizzato alla gestione integrata delle attività di un agriturismo, con particolare focus sulla clientela e sui servizi offerti.

La memorizzazione di nome, cognome, codice fiscale, indirizzo email e numero di telefono degli utenti è propedeutica alla creazione di un account individuale, 
al quale vengono associate credenziali di accesso univoche. 
\\L'autenticazione tramite tali credenziali è una condizione necessaria per l'interazione con la piattaforma di prenotazione, 
garantendo che ogni operazione sia riconducibile a un cliente specifico e registrato.

Il sistema permette di gestire un catalogo di servizi diversificato, ognuno dotato di parametri e vincoli operativi propri. 
Le prenotazioni possono riguardare il pernottamento in camere (con specifica della tipologia e del periodo di soggiorno), la riserva di tavoli al ristorante (indicando orario e numero di coperti) o l'accesso ai lettini in piscina (prenotabili per fascia oraria e posizione). 
Sono incluse anche attività ricreative, come le esperienze con animali (specificandone la tipologia) e la prenotazione del campo da calcio.

La granularità del sistema consente di applicare regole precise, come la durata fissa per alcuni noleggi o la disponibilità variabile in base all'occupazione di un determinato servizio. 
\\Ad esempio, i lettini in piscina sono prenotabili per fasce di 2 ore, mentre il campo da calcio può essere affittato per un'ora o più, a seconda della disponibilità.

Non è previsto solo l'acquisto di prestazioni singole, ma anche di pacchetti combinati che possono essere standardizzati dallo staff o personalizzati in base alle esigenze dei clienti. 
Questi pacchetti aggregano più servizi a una tariffa dedicata, soggetta a variazioni basate sulla stagionalità o su promozioni speciali attive.

L'agriturismo tiene traccia di tutte le prenotazioni effettuate, mantenendo uno storico che permette di analizzare le preferenze e di predisporre offerte personalizzate.

A completamento del ciclo di fruizione del servizio, i clienti hanno la facoltà di lasciare una recensione, articolata in un voto da 1 a 5 stelle e un campo testuale per eventuali commenti. 
\\Queste valutazioni rappresentano un prezioso strumento per il monitoraggio della qualità e l'ottimizzazione dell'offerta. Il sistema estende inoltre le sue funzionalità alla gestione di eventi speciali organizzati dalla struttura, permettendo di definirne le caratteristiche (capienza massima, costi, calendario) e di amministrare le iscrizioni direttamente dalla piattaforma.

Per il personale operativo, verrà sviluppata una dashboard che fornisce un monitoraggio in tempo reale di indicatori chiave, come lo stato delle prenotazioni e l'occupazione delle risorse. 
La piattaforma permetterà la generazione di report periodici per l'analisi dell'andamento delle prestazioni, delle preferenze della clientela e dell'efficacia delle promozioni. 
\\Inoltre, al fine di garantire la sicurezza e la corretta segmentazione delle responsabilità, 
il sistema implementa un controllo degli accessi basato sui ruoli, assegnando a ogni membro dello staff permessi specifici in linea con le proprie mansioni.

\section{Estrazione dei concetti principali}
\begin{center}
    \begin{tabularx}{\textwidth}{ l X r }
        \rowcolor{blue!30}
        \textbf{Termine}     & \textbf{Descrizione}                                                                                & \textbf{Sinonimi} \\
        \hline
        Opsite               & Persona che utilizza i servizi dell'agriturismo, registrata con dati anagrafici e contatti.         & Utente            \\
        \hline
        Prenotazione         & Richiesta di utilizzo di un servizio (camera, tavolo, lettino, attività) in un determinato periodo. & Prenotazione      \\
        \hline
        Camera               & Struttura ricettiva con tipologia (singola, doppia, suite) e capacità definita.                     & Alloggio          \\
        \hline
        Tavolo               & Postazione nel ristorante, con numero di posti e posizione (interno/esterno).                       & Tavolo            \\
        \hline
        Lettino              & Postazione nell'area piscina, prnotabile per fasce orarie.                                          & Ombrellone        \\
        \hline
        Attività con animali & Esperienza guidata con animali della fattoria didattica.                                            & Visita            \\
        \hline
        Campo da calcio      & Area sportiva prenotabile per parite e tornei.                                                      & Campo sportivo    \\
        \hline
        Pacchetto            & Combinazione di servizi con prezzo personalizzato.                                                  & Offerta           \\
        \hline
        Recensione           & Valutazione e commento lascaiti da un cliente su un servizio utilizzato.                            & Valutazione       \\
        \hline
        Amministratore       & Utente con i privilegi per gestire disponibilità, prenotiazioni e report.                           & Staff             \\
        \hline
        Evento               & Attività speciale organizzate (tornei, cene a tema).                                                & Attività          \\
        \hline
        Menu                 & Proposta gastronimica del ristorante/bar, con allergeni e stagionalità.                             & Menu              \\
        \hline
        Disponibilità        & Sato temporale e quantitativo di un servizio.                                                       & Capacità residua  \\
        \hline
        Abbonamento          & Formula di accesso ripetuto a servizi                                                               & Mermbership       \\
    \end{tabularx}
\end{center}

\begin{quote}
Il modello concettuale del sistema si articola attorno ai seguenti elementi chiave:

\begin{itemize}
    \item \textbf{Cliente:} L'entità centrale del sistema. Per ogni cliente vengono registrati i dati anagrafici (nome, cognome, codice fiscale) e di contatto (email, telefono), associati a un account personale per l'accesso e le prenotazioni.

    \item \textbf{Servizi e Prenotazioni:} Il cliente autenticato può effettuare prenotazioni per i diversi servizi offerti, che costituiscono il nucleo operativo della piattaforma. Questi includono:
    \begin{itemize}
        \item \textbf{Camere:} Differenziate per tipologia (es. singola, doppia, suite) e associate a una tariffa notturna.
        \item \textbf{Tavoli al Ristorante:} Prenotabili specificando il numero di coperti e l'orario.
        \item \textbf{Lettini in Piscina:} Disponibili per fasce orarie predefinite.
        \item \textbf{Attività con Animali:} Come visite guidate alla fattoria didattica.
        \item \textbf{Campo da Calcio:} Disponibile per slot orari predefiniti.
    \end{itemize}

    \item \textbf{Pacchetti Combinati:} Oltre ai servizi singoli, il sistema gestisce pacchetti che aggregano più prestazioni a un prezzo convenzionato. Possono essere standard (creati dallo staff) o personalizzati.

    \item \textbf{Recensioni:} Dopo la fruizione di un servizio, il cliente può lasciare una valutazione, tipicamente con un voto numerico (es. 1-5 stelle), e un commento testuale.

    \item \textbf{Staff ed Eventi:} Il personale, gestito con account e ruoli specifici, utilizza la piattaforma per amministrare la disponibilità dei servizi e organizzare eventi speciali (es. tornei, cene a tema).
\end{itemize}

 Particolare attenzione verrò posta sulla gestione dei \textbf{Vincoli operativi}, che definiscono le logiche e le limitazioni specifiche di ogni servizio, come la capacità massima, gli orari e le politiche di prenotazione.
\end{quote}

\chapter{Preogettazione concettuale}
\section{Schema scheletro}
\section{Raffinamenti proposti}

\section{Schema concettuale finale}


\chapter{Progettazione logica}
\section{Stima del volume dei dati}

\section{Descrizione delle operazioni principali e stima della loro frequenza}

\section{Schemi di navigazione e tabelle degli accessi}

\section{Raffinamento dello schema}

\section{Analisi delle ridondanze}

\section{Traduzione di entità e associazioni in relazioni}

\section{Schema relazionale finale}

\section{Traduzione delle operazioni in query SQL}


\chapter{Progettazione dell'applicazione}


\end{document}
