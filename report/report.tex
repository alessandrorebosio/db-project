\documentclass[a4paper,11pt]{report}

\usepackage{alltt, fancyvrb, url}
\usepackage{graphicx}
\usepackage[utf8]{inputenc}
\usepackage{float}
\usepackage[table]{xcolor}
\usepackage{hyperref}
\usepackage{tabularx}
\usepackage[top=2cm, bottom=2cm, left=2cm, right=2cm]{geometry}
% \usepackage{showframe}

% Questo commentalo se vuoi scrivere in inglese.
\usepackage[italian]{babel}

\usepackage[italian]{cleveref}

\title{Elaborato per il corso di\\''Basi di Dati''}
\author{
    Maisam Noumi \\
    Alessandro Rebosio (alessandro.rebosio@studio.unibo.it)\\
    Filippo Ricciotti
}
\date{\today}


\begin{document}


\maketitle

\tableofcontents

\chapter{Analisi dei requisiti}
Si ha come obiettivo la realizzazione di un database per la gestione di un agriturismo che offre un'ampia gamma di servizi: ristorazione,
ospitalità, piscina, attività con animali, sport e molto altro.

\section{Intervista}
Il sistema che si vuole realizzare è finalizzato alla gestione integrata delle attività di un agriturismo, con particolare focus sulla clientela e sui servizi offerti.

La memorizzazione di nome, cognome, codice fiscale, indirizzo email e numero di telefono degli utenti è propedeutica alla creazione di un account individuale, 
al quale vengono associate credenziali di accesso univoche. 
\\L'autenticazione tramite tali credenziali è una condizione necessaria per l'interazione con la piattaforma di prenotazione, 
garantendo che ogni operazione sia riconducibile a un cliente specifico e registrato.

Il sistema permette di gestire un catalogo di servizi diversificato, ognuno dotato di parametri e vincoli operativi propri. 
Le prenotazioni possono riguardare il pernottamento in camere (con specifica della tipologia e del periodo di soggiorno), la riserva di tavoli al ristorante (indicando orario e numero di coperti) o l'accesso ai lettini in piscina (prenotabili per fascia oraria e posizione). 
Sono incluse anche attività ricreative, come le esperienze con animali (specificandone la tipologia) e la prenotazione del campo da calcio.

La granularità del sistema consente di applicare regole precise, come la durata fissa per alcuni noleggi o la disponibilità variabile in base all'occupazione di un determinato servizio. 
\\Ad esempio, i lettini in piscina sono prenotabili per fasce di 2 ore, mentre il campo da calcio può essere affittato per un'ora o più, a seconda della disponibilità.

Non è previsto solo l'acquisto di prestazioni singole, ma anche di pacchetti combinati che possono essere standardizzati dallo staff o personalizzati in base alle esigenze dei clienti. 
Questi pacchetti aggregano più servizi a una tariffa dedicata, soggetta a variazioni basate sulla stagionalità o su promozioni speciali attive.

L'agriturismo tiene traccia di tutte le prenotazioni effettuate, mantenendo uno storico che permette di analizzare le preferenze e di predisporre offerte personalizzate.

A completamento del ciclo di fruizione del servizio, i clienti hanno la facoltà di lasciare una recensione, articolata in un voto da 1 a 5 stelle e un campo testuale per eventuali commenti. 
\\Queste valutazioni rappresentano un prezioso strumento per il monitoraggio della qualità e l'ottimizzazione dell'offerta. Il sistema estende inoltre le sue funzionalità alla gestione di eventi speciali organizzati dalla struttura, permettendo di definirne le caratteristiche (capienza massima, costi, calendario) e di amministrare le iscrizioni direttamente dalla piattaforma.

Per il personale operativo, verrà sviluppata una dashboard che fornisce un monitoraggio in tempo reale di indicatori chiave, come lo stato delle prenotazioni e l'occupazione delle risorse. 
La piattaforma permetterà la generazione di report periodici per l'analisi dell'andamento delle prestazioni, delle preferenze della clientela e dell'efficacia delle promozioni. 
\\Inoltre, al fine di garantire la sicurezza e la corretta segmentazione delle responsabilità, 
il sistema implementa un controllo degli accessi basato sui ruoli, assegnando a ogni membro dello staff permessi specifici in linea con le proprie mansioni.

\section{Estrazione dei Concetti Principali}
% La seguente riga aumenta lo spazio verticale tra le righe
\renewcommand{\arraystretch}{1.5} 
\begin{center}
% Prima e terza colonna la definisco con "l" per allinearne il contenuto a sinistra. 
% Quella centrale con "X" per consentire alla colonna di espandersi per riempire lo spazio disponibile 
    \begin{tabularx}{\textwidth}{l >{\raggedright\arraybackslash}X l}
        \rowcolor{blue!25}
        \textbf{Termine } &                         \textbf{Descrizione} &                          \textbf{Sinonimi } \\
        \hline
        \textbf{Cliente} & Persona che utilizza i servizi dell'agriturismo, registrata con dati anagrafici e di contatto. & Ospite, Utente \\
        \hline
        \textbf{Prenotazione} & Richiesta per l'utilizzo di un servizio specifico (camera, tavolo, ecc.) in una data e ora definite. & Riservazione \\
        \hline
        \textbf{Camera} & Unità ricettiva per il soggiorno, caratterizzata da tipologia (es. singola, doppia) e capacità. & Alloggio \\
        \hline
        \textbf{Tavolo} & Postazione nel ristorante definita da un numero di coperti e da una posizione specifica. & Postazione \\
        \hline
        \textbf{Lettino} & Area attrezzata in piscina (es. con lettino e ombrellone), prenotabile per fasce orarie. & Lettino, Ombrellone \\
        \hline
        \textbf{Attività Guidata} & Esperienza organizzata, come una visita alla fattoria didattica o un laboratorio. & Esperienza, Visita \\
        \hline
        \textbf{Campo da Calcio} & Area sportiva prenotabile per in fasce orarie per partite e tornei di calcio. & Campo da gioco \\
        \hline
        \textbf{Pacchetto Combinato} & Combinazione di più servizi offerti a un prezzo speciale. & Offerta, Pacchetto \\
        \hline
        \textbf{Recensione} & Feedback lasciato da un cliente su un servizio, composto da un voto e un commento testuale. & Valutazione, Feedback \\
        \hline
        \textbf{Staff} & Personale dell'agriturismo con specifici ruoli e permessi per la gestione del sistema. & Amministratore, Gestore, Staff \\
        \hline
        \textbf{Evento} & Attività speciale con date e capienza definite (es. tornei, cene a tema). & Attività \\
        \hline
        \textbf{Menù} & Proposta gastronomica del ristorante, con dettagli su piatti, prezzi, allergeni e stagionalità. & Menù \\
        \hline
        \textbf{Disponibilità} & Stato di un servizio in un dato momento, che ne definisce la fruibilità in termini quantitativi e temporali. & Capacità Residua \\
        \hline
        \textbf{Abbonamento} & Formula di accesso periodico o a ingressi multipli per uno o più servizi (es. piscina). & Membership \\
    \end{tabularx}
\end{center}


\section{Sintesi dei concetti}
\begin{quote}
Il modello concettuale del sistema si articola attorno a diversi elementi chiave, interconnessi 
per creare una piattaforma gestionale completa. \\[6pt]
Più nello specifico, al centro del sistema si trova il \textbf{Cliente}, per il quale vengono registrati i dati anagrafici e di contatto 
(nome, cognome, codice fiscale, email, telefono), associati a un account personale che ne abilita l'accesso e 
la capacità di effettuare prenotazioni. \\[6pt]
Una volta autenticato, il cliente può effettuare \textbf{prenotazioni} per un'ampia gamma di \textbf{servizi} 
che costituiscono il nucleo operativo della piattaforma: dalle \textbf{camere}, differenziate per tipologia 
e tariffa notturna, ai \textbf{tavoli al ristorante}, prenotabili per orario e numero di coperti; 
dai \textbf{lettini in piscina}, disponibili per fasce orarie predefinite, fino alle \textbf{attività con animali}, 
come le visite guidate, e all'utilizzo del \textbf{campo da calcio} per slot orari. \\[6pt]
Oltre ai servizi singoli, il sistema gestisce anche \textbf{pacchetti combinati}, 
che aggregano più prestazioni a un prezzo convenzionato e possono essere sia standard, 
creati dallo staff, sia personalizzati. \\[6pt]
A seguito della fruizione di un servizio, è prevista la possibilità per il cliente di lasciare una \textbf{recensione},
composta da una valutazione numerica e un commento testuale, al fine di monitorare la qualità dell'offerta. \\[6pt]
La gestione da parte del personale è supportata da account dedicati con ruoli specifici,
che permettono allo \textbf{staff} non solo di amministrare la disponibilità dei servizi, 
ma anche di organizzare \textbf{eventi} speciali come tornei o cene a tema. \\[6pt]
Particolare attenzione verrà posta sulla gestione dei \textbf{vincoli operativi}, 
i quali definiscono le logiche e le limitazioni specifiche di ogni servizio (come la capacità massima, gli orari e le politiche di prenotazione) garantendo la coerenza e il corretto funzionamento dell'intero sistema.
\end{quote}



\chapter{Preogettazione concettuale}
\section{Schema scheletro}
\section{Raffinamenti proposti}

\section{Schema concettuale finale}


\chapter{Progettazione logica}
\section{Stima del volume dei dati}

\section{Descrizione delle operazioni principali e stima della loro frequenza}

\section{Schemi di navigazione e tabelle degli accessi}

\section{Raffinamento dello schema}

\section{Analisi delle ridondanze}

\section{Traduzione di entità e associazioni in relazioni}

\section{Schema relazionale finale}

\section{Traduzione delle operazioni in query SQL}


\chapter{Progettazione dell'applicazione}


\end{document}
