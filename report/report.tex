\documentclass[a4paper,11pt]{report}

\usepackage{alltt, fancyvrb, url}
\usepackage{graphicx}
\usepackage[utf8]{inputenc}
\usepackage{float}
\usepackage{xcolor}
\usepackage{hyperref}
\usepackage[a4paper, top=2cm, bottom=2cm, left=2cm, right=2cm]{geometry}

% Questo commentalo se vuoi scrivere in inglese.
\usepackage[italian]{babel}

\usepackage[italian]{cleveref}

\title{Elaborato per il corso di\\''Basi di Dati''}
\author{
    Maisam Noumi\\
    Alessandro Rebosio\\
    Filippo Ricciotti
}
\date{\today}


\begin{document}


\maketitle

\tableofcontents

\chapter{Analisi dei requisiti}
Si ha come obiettivo la realizzazione di un database per la gestione di un agriturismo che offre un'ampia gamma di servizi: ristorazione,
ospitalità, piscina, attività con animali, sport e molto altro.

\section{Intervista}
Si vuole tenere traccia dei clienti dell'agriturismo memorizzandone nome, cognome, codice fiscale, indirizzo email e numero di telefono. Al
momento della registrazione, viene creato un account personale con credenziali di accesso per la piattaforma. Ogni cliente
può effettuare prenotazioni solo se autenticato, e ogni prenotazione viene registrata con i relativi dettagli (servizio richiesto, data,
orario e numero di partecipanti).

Le prenotazioni possono riguardare diversi servizi: camere (con specifica della tipologia e del periodo di soggiorno), tavoli al
ristorante (indicando l'orario e il numero di coperti), lettini in piscina (con fascia oraria e posizione), attività con animali
(specificando il tipo di esperienza) e campo da calcio (indicando la durata e l'eventuale partecipazione a tornei). Ogni servizio ha
regole specifiche: ad esempio, i lettini in piscina sono prenotabili per fasce di 2 ore, mentre il campo da calcio può essere prenotato
per un'ora o più, a seconda della disponibilità.

I clienti possono acquistare sia servizi singoli che pacchetti combinati, creati dallo staff o personalizzati in base alle esigenze. I pacchetti
possono includere più servizi (es. pernottamento + cena + attività con animali) e hanno un prezzo specifico, con eventuali sconti applicabili
in base alla stagione o a promozioni speciali. Ogni pacchetto può essere modificato dallo staff per adattarlo a particolari esigenze dei clienti.

Ogni prenotazione viene confermata via email o SMS, con la possibilità di modificarla o cancellarla entro un certo termine. In caso di
mancata presentazione senza preavviso, può essere applicata una penale. Le prenotazioni per eventi speciali (come tornei o cene a tema) richiedono
spesso un acconto non rimborsabile.

L'agriturismo tiene traccia di tutte le prenotazioni effettuate dai clienti, mantenendo uno storico che permette di analizzare le preferenze e di
proporre offerte personalizzate. I clienti possono lasciare recensioni per ogni servizio utilizzato, valutandolo con un voto da 1 a 5
stelle e aggiungendo un commento. Le recensioni vengono moderate dallo staff e possono essere utilizzate per migliorare i servizi offerti.

Il sistema permette anche la gestione degli eventi organizzati dall'agriturismo, come tornei di calcio o laboratori con animali. Per ogni
evento è possibile definire il numero massimo di partecipanti, il prezzo e le date disponibili. I clienti possono iscriversi agli
eventi direttamente dalla piattaforma, ricevendo una conferma e i dettagli organizzativi.

Lo staff dell'agriturismo ha accesso a una dashboard che mostra in tempo reale lo stato delle prenotazioni, l'occupazione delle
camere e dei servizi, e le recensioni ricevute. Possono generare report periodici per analizzare l'andamento delle prenotazioni, le
preferenze dei clienti e l'efficacia delle promozioni. Inoltre, il sistema supporta la gestione di account multipli per lo staff, con
diversi livelli di accesso in base al ruolo (es. receptionist, responsabile del ristorante, gestore della piscina).

\section{Rilevamento delle ambiguità e correzioni proposte}

\section{Definizione delle specifiche in linguaggio naturale ed estrazione dei concetti principali}


\chapter{Preogettazione concettuale}
\section{Schema scheletro}
\section{Raffinamenti proposti}

\section{Schema concettuale finale}


\chapter{Progettazione logica}
\section{Stima del volume dei dati}

\section{Descrizione delle operazioni principali e stima della loro frequenza}

\section{Schemi di navigazione e tabelle degli accessi}

\section{Raffinamento dello schema}

\section{Analisi delle ridondanze}

\section{Traduzione di entità e associazioni in relazioni}

\section{Schema relazionale finale}

\section{Traduzione delle operazioni in query SQL}


\chapter{Progettazione dell'applicazione}


\end{document}
